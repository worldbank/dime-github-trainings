% ---------------------------- Preamble starts here ----------------------------

\documentclass[aspectratio=169]{beamer} %Remove [aspectratio=169] to get non-wide 4:3 slide aspect ratio

%-----------------------------------------------
% --- Set beamer theme
\usetheme{Metropolis}
\setbeamertemplate{footline}{}				% Remove automatic footer
\setbeamertemplate{navigation symbols}{}	% Comment this line to display navigation symbols

%-----------------------------------------------
% Load i2i symbol
\addtobeamertemplate{frametitle}{}{%
\begin{textblock*}{\linewidth}(0cm,7.4cm) % Replace with (0cm, 8cm) if using non-wide slide aspect
	\includegraphics[width=\linewidth]{../../Common-Resources/img/Footer.png}
\end{textblock*}}

\setbeamertemplate{footline}{\hfill\insertframenumber/\inserttotalframenumber}

%-----------------------------------------------
% --- Load packages
\usepackage{textpos}		% To align objects correctly
\usepackage{multicol}		% To right in multiple columns
\usepackage{color}			% To color text

%-----------------------------------------------
% --- Include link to last commit
\usepackage{xstring}
\usepackage{catchfile}

%Set this user input
\newcommand{\gitfolder}{../../../.git} %relative path to .git folder from .tex doc
\newcommand{\reponame}{worldbank/dime-github-trainings} % Name of account and repo be set in URL

%Based on this https://tex.stackexchange.com/questions/455396/how-to-include-the-current-git-commit-id-and-branch-in-my-document
\CatchFileDef{\headfull}{\gitfolder/HEAD.}{} 				%Get path to head file for checked out branch
\StrGobbleRight{\headfull}{1}[\head]						%Remove end of line character
\StrBehind[2]{\head}{/}[\branch]							%Parse out the path only
\CatchFileDef{\commit}{\gitfolder/refs/heads/\branch.}{}	%Get the content of the branch head
\StrGobbleRight{\commit}{1}[\commithash]					%Remove end of line characted

%Build the URL to this commit based on the information we now have
\newcommand{\commiturl}{\url{https://github.com/\reponame/commit/\commithash}}

%-----------------------------------------------
% --- Add your information here
\title{GitHub - Pull Request training}
\author{DIME Analytics}
\institute{DIME - The World Bank - \trainingURL{https://www.worldbank.org/en/research/dime}}
\date{\today}

\newcommand{\trainingURL}[1]{{\color{blue}\url{#1}}}

\newcommand{\traininerUsername}{kbjarkefur}
\newcommand{\repoName}{\traininerUsername/lyrics}
\newcommand{\trainingRepoURL}[1]{\trainingURL{github.com/\repoName #1}}
\newcommand{\trainerEmail}{\trainingURL{kbjarkefur@worldbank.org} }


% ---------------------------- Preamble ends here ----------------------------

\begin{document}

\begin{frame}
\includegraphics[width=\textwidth]{../../Common-Resources/img/Header.png}
\vspace{-0.2cm}
\titlepage 	 % Opening slide, prints inform
\end{frame}

\section{Intro - Pull Request training}

\begin{frame}
	\frametitle{Prerequisites}
	\begin{columns}[c]

		\column{.65\textwidth} % Left column and width

		\large \textbf{Who is this training for?}

		\vspace{1em}

		Anyone comfortable branching, committing and merging in Git/GitHub
		who is ready to be introduced to a more advanced workflow
		using Git/GitHub to ensure high quality code.

		\column{.25\textwidth} % Left column and width

		\begin{minipage}[t][3.5cm][t]{\textwidth}
			\begin{figure}
				\centering
				\includegraphics[width=.8\textwidth]{./img/git-icon.png}
			\end{figure}
		\end{minipage}
		\vspace{-.5cm}
		\begin{minipage}[t][3.5cm][t]{\textwidth}
			\begin{figure}
				\centering
				\includegraphics[width=.8\textwidth]{./img/github-icon.png}
			\end{figure}
		\end{minipage}

		\column{.05\textwidth} % Left column and width

	\end{columns}
\end{frame}

\begin{frame}
	\frametitle{Content}

	\Large\centering \textbf{THIS TRAINING HAS THREE PARTS}

	\vspace{1.2em}

	\raggedright
	\begin{columns}[T]

		\column{.01\textwidth}
		\column{.3\textwidth} % Left column and width	
			\large \textbf{Part 1}
			
			\vspace{1em}
			\raggedright
			\normalfont A very brief review of how to work with branches in Git

		\column{.01\textwidth}
		\column{.3\textwidth} % Left column and width
			\large \textbf{Part 2}
			
			\vspace{1em}
			\raggedright
			\normalfont Introducing the \textit{branch-PR-merge} cycle for how to use the review features in a PR
			
		\column{.01\textwidth}
		\column{.3\textwidth} % Left column and width
			\large \textbf{Part 3}
			
			\vspace{1em}
			\raggedright
			\normalfont Gitflow - a philosophy for how to work with branches and merge/PRs in Git
		
		\column{.01\textwidth}

	\end{columns}
\end{frame}

\section{Part 1: \newline A brief intro to working with branches}

\begin{frame}
	\frametitle{Brief intro to branches}
	\begin{columns}[c]
		
		\column{.68\textwidth} % Left column and width
		
		Branches is the killer feature of Git. 
		It allows us to approach tasks in a non-linear way,
		which is similar to how human brains
		and especially teams of humans work.
		\vspace{1.5em}	

		It's thanks to branches we can - without creating conflicts -
		test multiple solutions to as task at the same time, or 
		have multiple people working at different tasks at the same time.
		\vspace{1.5em}
			
		Branches are often visualized in network graphs.
		
		\column{.25\textwidth} % Left column and width
		
		\begin{minipage}[t][6.5cm][t]{\textwidth}
			\begin{figure}
				\centering
				\includegraphics[width=.55\textwidth]{./img/pr-branch.png}
			\end{figure}
		\end{minipage}
		
		\column{.05\textwidth} % Left column and width
		
	\end{columns}
\end{frame}


\begin{frame}
	\frametitle{What are pull request?}
	\begin{columns}[c]
		
		\column{.68\textwidth} % Left column and width
		
		\textit{Pull requests} (PRs) are part of GitHub, but not part of Git.	
		\vspace{2em}
		
		GitHub is to Git, what Gmail is to email.
		\vspace{2em}
		
		A PR is a request to merge one branch into another,
		and adds great QA and review features before the merge.
		\column{.25\textwidth} % Left column and width
		
		\begin{minipage}[t][6.5cm][t]{\textwidth}
			\begin{figure}
				\centering
				\includegraphics[width=.55\textwidth]{./img/pr-highlight.png}
			\end{figure}
		\end{minipage}
		
		\column{.05\textwidth} % Left column and width
		
	\end{columns}
\end{frame}

\section{Part 2: \newline Best practices for the branch-PR-merge cycle}

\begin{frame}
\frametitle{The branch-PR-merge cycle}

	\begin{columns}[c]

		\column{.35\textwidth} % Left column and width
		\large Three stages with multiple steps in each
		\vspace{.7cm}\newline
		\large Complete a full \textit{branch-PR-merge} cycle for \textbf{all tasks}
		\vspace{.7cm}\newline
		\large We will go over all steps but
		focus today on the \textbf{work stage}

		\column{.65\textwidth} % Right column and width
		\vspace{-.75cm}
		\begin{figure}
			\centering
			\includegraphics[width=\textwidth]{./img/branch-pr-merge-cycle.png}
		\end{figure}

	\end{columns}
\end{frame}

\begin{frame}
	\frametitle{Stage: The set up stage}

	\huge\centering \textbf{SET UP STAGE}

\end{frame}


\begin{frame}
	\frametitle{Stage: The set up stage}
	\begin{columns}[c]

		\column{.35\textwidth} % Left column and width

		\Large \textbf{Purpose of the set up stage:}
		\vspace{1em}
		\normalsize

		\begin{itemize}
			\setlength\itemsep{1em}
			\item Document the task you will work on
			\item Create a space in your repo to work on the task
		\end{itemize}

		\column{.65\textwidth} % Right column and width
		\vspace{-.75cm}
		\begin{figure}
			\centering
			\includegraphics[width=\textwidth]{./img/branch-pr-merge-cycle-S1.png}
		\end{figure}

	\end{columns}
\end{frame}


\begin{frame}
	\frametitle{Step: Create an issue}
	\begin{columns}[c]

		\column{.35\textwidth} % Left column and width
		\begin{itemize}
			\setlength\itemsep{.5em}
			\item Issues is a far superior way to document tasks
			compared to emails or in-person meetings!
			\item They allow other team members to provide input
			on how to solve the task
			\item Very quick tasks that do not require discussion may not need an issue
		\end{itemize}

		\column{.65\textwidth} % Right column and width
		\vspace{-.75cm}
		\begin{figure}
			\centering
			\includegraphics[width=\textwidth]{./img/branch-pr-merge-cycle-S1-1.png}
		\end{figure}

	\end{columns}
\end{frame}

\begin{frame}
	\frametitle{Create an issue}
	\begin{columns}[c]

		\column{.55\textwidth} % Right column and width
		\vspace{-.5cm}
		\begin{figure}
			\centering
			\includegraphics[width=\textwidth]{./img/create-issue-1.png}
		\end{figure}

		\column{.45\textwidth} % Left column and width

		\begin{itemize}
			\setlength\itemsep{.5em}
			\item Click the issues tab and then \texttt{New Issue}
			\item Describe the task and assign yourself or someone else to complete the task
			\item Ask the rest of the team of input if applicable
			\item Add labels to facilitate future searches

		\end{itemize}

	\end{columns}
\end{frame}

\begin{frame}
	\frametitle{Browse an issue}
	\begin{columns}[c]

		\column{.55\textwidth} % Right column and width
		\vspace{-.5cm}
		\begin{figure}
			\centering
			\includegraphics[width=\textwidth]{./img/create-issue-2.png}
		\end{figure}

		\column{.45\textwidth} % Left column and width

		\begin{itemize}
			\setlength\itemsep{1em}
			\item Take note of the issue number
			(also found in end of issue URL)
			so you can refer to it
			\item Anyone can provide input using the comment field
			\item Assignees and labels can be updated as the task evolve
			\item Future team members can browse closed issues
		\end{itemize}

	\end{columns}
\end{frame}

\begin{frame}
	\frametitle{Create a branch}
	\begin{columns}[c]

		\column{.35\textwidth} % Left column and width
		\begin{itemize}
			\setlength\itemsep{.5em}
			\item Create branch on GitHub.com
			\hyperlink{new-branch}{\beamergotobutton{Step-by-step}}
			\item We create a branch to have a space
			where work on this task does not interfere
			with work on other tasks
			\item Name the branch after the task and
			add the issue number as suffix - for example:
			\texttt{pr-training-11}
		\end{itemize}

		\column{.65\textwidth} % Right column and width
		\vspace{-.75cm}
		\begin{figure}
			\centering
			\includegraphics[width=\textwidth]{./img/branch-pr-merge-cycle-S1-2.png}
		\end{figure}

	\end{columns}
\end{frame}

\begin{frame}
	\frametitle{Open the PR}
	\begin{columns}[c]

		\column{.35\textwidth} % Left column and width
		\begin{itemize}
			\setlength\itemsep{1em}
			\item Create a PR \textbf{before} you implement the task
			- this creates a space on GitHub.com
			where the progress of this task can be followed
			\item You can create a \textit{``draft PR''} 
			that cannot be merged while in draft status
			- signals work in progress
		\end{itemize}

		\column{.65\textwidth} % Right column and width
		\vspace{-.75cm}
		\begin{figure}
			\centering
			\includegraphics[width=\textwidth]{./img/branch-pr-merge-cycle-S1-3.png}
		\end{figure}

	\end{columns}
\end{frame}

\begin{frame}
	\frametitle{Create a PR}
	\begin{columns}[c]

		\column{.53\textwidth} % Right column and width
		\vspace{-.5cm}
		\begin{figure}
			\centering
			\includegraphics[width=\textwidth]{./img/create-pr-0.png}
		\end{figure}

		\column{.47\textwidth} % Left column and width
		Recap of PR creation best practices:
		\vspace{.5cm}
		\begin{itemize}
			\setlength\itemsep{.5em}
			\item Go to the \textit{``Pull request''} tab
			\item Make sure that you are comparing the correct branches
			\item Always browse the commits and
			the files changes before opening the PR
			\item You need at least one commit
			- Create/edit a README describing the task
		\end{itemize}

	\end{columns}
\end{frame}

\begin{frame}
	\frametitle{Create a draft PR}
	\begin{columns}[c]

		\column{.53\textwidth} % Right column and width
		\vspace{-.5cm}
		\begin{figure}
			\centering
			\includegraphics[width=\textwidth]{./img/create-pr-1.png}
		\end{figure}

		\column{.47\textwidth} % Left column and width

		\begin{itemize}
			\setlength\itemsep{.5em}
			\item Give the PR a name associated with
			the branch name but more descriptive
			\item Add \texttt{\#11} in the description
			to create a hyper link to issue
			- add \texttt{work-in-progress} label
			\item Click the arrow on the green button and
			select \textit{Create draft pull request}
		\end{itemize}

	\end{columns}
\end{frame}

\begin{frame}
	\frametitle{Browse draft PR}
	\begin{columns}[c]

		\column{.6\textwidth} % Right column and width
		\vspace{-.5cm}
		\begin{figure}
			\centering
			\includegraphics[width=\textwidth]{./img/create-pr-2.png}
		\end{figure}

		\column{.4\textwidth} % Left column and width

		\begin{itemize}
			\setlength\itemsep{1em}
			\item A draft PR has the color code gray
			\item You can follow someone's work here
			- but no one is required to review anything yet
			\item The author will eventually remove the draft status
			and ask for reviews
		\end{itemize}

	\end{columns}
\end{frame}

\begin{frame}
	\frametitle{Stage: The work stage}

	\huge\centering \textbf{WORK STAGE}

\end{frame}

\begin{frame}
	\frametitle{Stage: The work stage}
	\begin{columns}[c]

		\column{.35\textwidth} % Left column and width

		\Large \textbf{Purpose of the work stage:}
		\vspace{.5em}
		\normalsize
		\begin{itemize}
			\setlength\itemsep{.5em}
			\item Write the code that implements this task
			\item Use the review features in a PR to improve code quality
			\item Iterate this stage until you have high-quality code
		\end{itemize}

		\column{.65\textwidth} % Right column and width
		\vspace{-.75cm}
		\begin{figure}
			\centering
			\includegraphics[width=\textwidth]{./img/branch-pr-merge-cycle-S2.png}
		\end{figure}

	\end{columns}
\end{frame}

\begin{frame}
	\frametitle{Make commits}
	\begin{columns}[c]

		\column{.35\textwidth} % Left column and width
		\begin{itemize}
			\setlength\itemsep{1em}
			\item The author commits edits as usual
			\item The draft PR will be updated as
			more commits are pushed to the branch
			\item Keep pushing commits until
			the first implementation of the task is complete
		\end{itemize}

		\column{.65\textwidth} % Right column and width
		\vspace{-.75cm}
		\begin{figure}
			\centering
			\includegraphics[width=\textwidth]{./img/branch-pr-merge-cycle-S2-1.png}
		\end{figure}

	\end{columns}
\end{frame}

\begin{frame}
	\frametitle{Un-draft the PR}
	\begin{columns}[c]

		\column{.35\textwidth} % Left column and width
		\begin{itemize}
			\setlength\itemsep{1em}
			\item Document what has been implemented in the branch
			\item Un-drafting indicates that a PR is ready to review
		\end{itemize}

		\column{.65\textwidth} % Right column and width
		\vspace{-.75cm}
		\begin{figure}
			\centering
			\includegraphics[width=\textwidth]{./img/branch-pr-merge-cycle-S2-2.png}
		\end{figure}

	\end{columns}
\end{frame}

\begin{frame}
	\frametitle{Un-draft the PR}
	\begin{columns}[c]

	\column{.6\textwidth} % Right column and width
	\vspace{-.75cm}
	\begin{figure}
		\centering
		\includegraphics[width=\textwidth]{./img/undraft-pr.png}
	\end{figure}

	\column{.4\textwidth} % Left column and width
	\begin{itemize}
		\setlength\itemsep{1em}
		\item Replace the \textit{work-in-progress} label
		with \textit{review-me}
		\item Assign one or several people to review your PR
		\item Notify the reviewer(s) using email
		or slack as GitHub notifications may be muted
		\item Click \textit{``Ready for review''} to remove draft status
	\end{itemize}

\end{columns}
\end{frame}

\begin{frame}
	\frametitle{Review the PR}
	\begin{columns}[c]

		\column{.35\textwidth} % Left column and width
		\begin{itemize}
			\setlength\itemsep{1em}
			\item PR reviews are incredibly important for high-quality code
			\item Use the PR to review code edits
			in the \textit{file changes} tab
			\item The reviewer adds code suggestions and comments
		\end{itemize}

		\column{.65\textwidth} % Right column and width
		\vspace{-.75cm}
		\begin{figure}
			\centering
			\includegraphics[width=\textwidth]{./img/branch-pr-merge-cycle-S2-3.png}
		\end{figure}

	\end{columns}
\end{frame}

\begin{frame}
	\frametitle{Review the PR}
	\begin{columns}[c]

		\column{.65\textwidth} % Right column and width
		\vspace{-.75cm}
		\begin{figure}
			\centering
			\includegraphics[width=\textwidth]{./img/review-1.png}
		\end{figure}

		\column{.35\textwidth} % Left column and width
		\begin{itemize}
			\setlength\itemsep{2em}
			\item Start your review by reading what the author or anyone else
			have already said in the \textit{``Conversation"} tab
			\item Then go to the \textit{``Files changed''} tab
			- this is where you will review the new content of the files
		\end{itemize}

	\end{columns}
\end{frame}


\begin{frame}
	\frametitle{Line comments}
	\begin{columns}[c]

		\column{.50\textwidth} % Right column and width
		\vspace{-.6cm}
		\begin{figure}
			\centering
			\includegraphics[width=\textwidth]{./img/line-comment-1.png}
		\end{figure}
		\vspace{-.3cm}
		\begin{figure}
			\centering
			\includegraphics[width=\textwidth]{./img/line-comment-2.png}
		\end{figure}

		\column{.50\textwidth} % Left column and width
		\begin{itemize}
			\setlength\itemsep{.74em}
			\item Make comments on specific lines of as much as possible
			- line comments are much easier to keep track of
			\item Hover with the mouse over the line number 
			for a line you want to comment on and click the plus sign 
			(blue circle) to open comment field
			\item Hold shift when click and drag mouse to comment multiple lines (black circle)
		\end{itemize}

	\end{columns}
\end{frame}

\begin{frame}
	\frametitle{Line suggestions}
	\begin{columns}[c]

		\column{.50\textwidth} % Right column and width
		\vspace{-.6cm}
		\begin{figure}
			\centering
			\includegraphics[width=\textwidth]{./img/suggestion-1.png}
		\end{figure}
		\vspace{-.3cm}
		\begin{figure}
			\centering
			\includegraphics[width=\textwidth]{./img/suggestion-2.png}
		\end{figure}

		\column{.50\textwidth} % Left column and width
		\begin{itemize}
			\setlength\itemsep{.74em}
			\item You can also suggest an edit directly to the code
			if the edit is not too big
			\item Click the suggestion button (red circle) 
			and edit the line of code in the comment field
			\item The author will then see the exact difference in your suggestion
			\item Suggestions can be done to multi-line comments too
			and can be combined with regular comments
		\end{itemize}

	\end{columns}
\end{frame}

\begin{frame}
	\frametitle{Overall comments}
	\begin{columns}[c]

		\column{.6\textwidth} % Right column and width
		\vspace{-.75cm}
		\begin{figure}
			\centering
			\includegraphics[width=\textwidth]{./img/pr-comment.png}
		\end{figure}

		\column{.4\textwidth} % Left column and width
		\begin{itemize}
			\setlength\itemsep{1em}
			\item Make overall comments and comments
			that cannot be done to a specific section of the code
			in thread in the \textit{``Conversation''} tab
			\item Tag the author so they get a notification
			that someone have reviewed the PR
			\item Suggestions are also listed in this tab
		\end{itemize}

	\end{columns}
\end{frame}

\begin{frame}
	\frametitle{Discuss and accept/reject}
	\begin{columns}[c]

		\column{.35\textwidth} % Left column and width
		\begin{itemize}
			\setlength\itemsep{1em}
			\item The author discusses, accepts or rejects suggestions
			and replies to comments
			\item The author, the reviewer(s)
			and/or anyone else in the project team then
			decide if the work stage needs to be repeated
		\end{itemize}

		\column{.65\textwidth} % Right column and width
		\vspace{-.75cm}
		\begin{figure}
			\centering
			\includegraphics[width=\textwidth]{./img/branch-pr-merge-cycle-S2-4.png}
		\end{figure}

	\end{columns}
\end{frame}

\begin{frame}
	\frametitle{Review suggestions}
	\begin{columns}[c]

		\column{.5\textwidth} % Right column and width
		\vspace{-.75cm}
		\begin{figure}
			\centering
			\includegraphics[width=\textwidth]{./img/review-suggestion-1.png}
		\end{figure}

		\column{.5\textwidth} % Left column and width
		\begin{itemize}
			\setlength\itemsep{1em}
			\item Respond to code suggestions before comments 
			as suggestions tend to be quicker to address
			\item Suggestions and line comments are listed chronologically
			in the \textit{``Conversation''} tab and
			by line number in the \textit{``File changes''} tab
			\item Accept the suggestion with \textit{Commit Suggestion},
			ignore/delete it with \textit{Resolve conversation}
			or comment to ask for more information
		\end{itemize}

	\end{columns}
\end{frame}

\begin{frame}
	\frametitle{Approve PR}
	\begin{columns}[c]

		\column{.35\textwidth} % Right column and width
		\begin{figure}
			\centering
			\includegraphics[width=.6\textwidth]{./img/qa.png}
		\end{figure}

		\column{.65\textwidth} % Left column and width
		\begin{itemize}
			\setlength\itemsep{1em}
			\item When all suggestions and comments
			have been responded to,
			the team should discuss if a task is done 
			or need more work 
			\item If major work is still required,
			revert the PR to draft status,
			add back the \textit{work-in-progress} label and
			iterate the work stage again
			\item Move on to the merge stage 
			only when the task is complete and 
			the team agrees the code is high-quality
		\end{itemize}

	\end{columns}
\end{frame}


\begin{frame}
	\frametitle{Stage: The merge stage}

	\huge\centering \textbf{MERGE STAGE}

\end{frame}

\begin{frame}
	\frametitle{Stage: The merge stage}
	\begin{columns}[c]

		\column{.35\textwidth} % Left column and width

		\Large \textbf{Purpose of the merge stage:}
		\vspace{1em}
		\normalsize
		\begin{itemize}
			\setlength\itemsep{.5em}
			\item Merge the quality controlled PR
			\item Make sure the task is properly documented and clean up the repo
		\end{itemize}

		\column{.65\textwidth} % Right column and width
		\vspace{-.75cm}
		\begin{figure}
			\centering
			\includegraphics[width=\textwidth]{./img/branch-pr-merge-cycle-S3.png}
		\end{figure}

	\end{columns}
\end{frame}


\begin{frame}
	\frametitle{Merge the PR}
	\begin{columns}[c]

		\column{.35\textwidth} % Left column and width
		\begin{itemize}
			\setlength\itemsep{.5em}
			\item Make sure that the PR is well documented
			- this is where future team members will read
			how something was done and why
			\item Then you can merge your well-reviewed and well-documented code
			\item \textbf{Test run your code after merging it!}

		\end{itemize}

		\column{.65\textwidth} % Right column and width
		\vspace{-.75cm}
		\begin{figure}
			\centering
			\includegraphics[width=\textwidth]{./img/branch-pr-merge-cycle-S3-1.png}
		\end{figure}

	\end{columns}
\end{frame}

\begin{frame}
	\frametitle{Delete the branch}
	\begin{columns}[c]

		\column{.35\textwidth} % Left column and width
		\begin{itemize}
			\setlength\itemsep{1em}
			\item Always delete branches after they are merged
			\item If you want to keep working in a branch with the same name,
			then re-create that branch
			\item Otherwise you risk working on an outdated version of the repo
		\end{itemize}

		\column{.65\textwidth} % Right column and width
		\vspace{-.75cm}
		\begin{figure}
			\centering
			\includegraphics[width=\textwidth]{./img/branch-pr-merge-cycle-S3-2.png}
		\end{figure}

	\end{columns}
\end{frame}

\begin{frame}
	\frametitle{Close issue}
	\begin{columns}[c]

		\column{.35\textwidth} % Left column and width
		If you created an issue for this task:
		\begin{itemize}
			\setlength\itemsep{.5em}
			\item Make sure that the PR number is referenced in the issue
			\item For example: \textit{``Issue resolved in PR \#12''}
			\item Then close the issue
		\end{itemize}

		\column{.65\textwidth} % Right column and width
		\vspace{-.75cm}
		\begin{figure}
			\centering
			\includegraphics[width=\textwidth]{./img/branch-pr-merge-cycle-S3-3.png}
		\end{figure}

	\end{columns}
\end{frame}



\section{Part 3: \newline Gitflow - How to fit the ``branch-PR-merge'' cycle into your workflow}

\begin{frame}
	\frametitle{The network graph}

	\vspace{-.5cm}
	\begin{minipage}[t][5cm][t]{\textwidth}
		\begin{figure}
			\centering
			\includegraphics[width=\textwidth]{./img/dime-gitflow-network.png}
		\end{figure}
	\end{minipage}

	\vspace{-.5cm}
	\begin{minipage}[t][5cm][t]{\textwidth}
		\begin{itemize}
			\setlength\itemsep{.5em}
			\item A typical git network graph - each dot is a commit
			\item Several branches have been opened and merged,
			but currently only the \textit{main} branch is open
		\end{itemize}
	\end{minipage}

\end{frame}

\begin{frame}
	\frametitle{Gitflow}
	\begin{columns}[c]

		\column{.65\textwidth} % Left column and width
		\begin{itemize}
			\setlength\itemsep{.5em}
			\item \textbf{Gitflow} is an idea or a philosophy 
			of how to organize work in git - it is not a software you install
			\item It is developed by computer scientists and
			you will find a lot of resources for it online
			\item We will present a simplified version adapted for research
			\item Two branch types (later we add a third type):
			\begin{itemize}
				\item The \textbf{main branch} (formerly master branch)
				- should never be worked in directly
				\item \textbf{Feature branches} - this is where you do all your work
			\end{itemize}
		\end{itemize}

		\column{.35\textwidth} % Right column and width
		\vspace{-.75cm}
		\begin{figure}
			\centering
			\includegraphics[width=.75\textwidth]{./img/organization.png}
		\end{figure}
	\end{columns}
\end{frame}



\begin{frame}
	\frametitle{Simple branch-PR-merge cycle}

	\huge\centering \textbf{Simple branch-PR-merge cycle}

\end{frame}


\begin{frame}
	\frametitle{Start a branch-PR-merge cycle}

	\vspace{-.5cm}
	\begin{minipage}[t][5cm][t]{\textwidth}
		\begin{figure}
			\centering
			\includegraphics[width=\textwidth]{./img/dime-gitflow-network-0.png}
		\end{figure}
	\end{minipage}

	\vspace{-.5cm}
	\begin{minipage}[t][5cm][t]{\textwidth}
		\begin{itemize}
			\setlength\itemsep{.5em}
			\item This can either be the beginning of a repo
			or at a point where all previous branches were merged
			to the main branch
			\item Let's say your next task is to set up folders
			- apply the \textit{branch-PR-merge} cycle
		\end{itemize}
	\end{minipage}

\end{frame}


\begin{frame}
	\frametitle{Set up stage - 1}

	\vspace{-.5cm}
	\begin{minipage}[t][5cm][t]{\textwidth}
		\begin{figure}
			\centering
			\includegraphics[width=\textwidth]{./img/dime-gitflow-network-1-0.png}
		\end{figure}
	\end{minipage}

	\vspace{-.5cm}
	\begin{minipage}[t][5cm][t]{\textwidth}
		\begin{itemize}
			\setlength\itemsep{.5em}
			\item Create a new \textit{feature} branch at the point
			where you want the the \textit{branch-PR-merge} cycle
			to start from
			\item Two branches pointing to the same commit
			are identical by definition
		\end{itemize}
	\end{minipage}

\end{frame}


\begin{frame}
	\frametitle{Set up stage - 2}

	\vspace{-.5cm}
	\begin{minipage}[t][5cm][t]{\textwidth}
		\begin{figure}
			\centering
			\includegraphics[width=\textwidth]{./img/dime-gitflow-network-1-1.png}
		\end{figure}
	\end{minipage}

	\vspace{-.5cm}
	\begin{minipage}[t][5cm][t]{\textwidth}
		\begin{itemize}
			\setlength\itemsep{.5em}
			\item Create a new short README or
			quickly edit an existing and commit -
			(or make a quick commit for the task) -
			at least one commit is needed to follow this workflow
			\item Open up a draft PR so there is a good place
			for your team to follow your work
		\end{itemize}
	\end{minipage}

\end{frame}

\begin{frame}
	\frametitle{Work stage}

	\vspace{-.5cm}
	\begin{minipage}[t][5cm][t]{\textwidth}
		\begin{figure}
			\centering
			\includegraphics[width=\textwidth]{./img/dime-gitflow-network-1-2.png}
		\end{figure}
	\end{minipage}

	\vspace{-.5cm}
	\begin{minipage}[t][5cm][t]{\textwidth}
		\begin{itemize}
			\setlength\itemsep{.5em}
			\item Complete the task using one or many commits
			- in this example only one more commit was needed
			\item Assign someone to review the PR
			- iterate until you are all satisfied
		\end{itemize}
	\end{minipage}

\end{frame}

\begin{frame}
	\frametitle{Merge stage}

	\vspace{-.5cm}
	\begin{minipage}[t][5cm][t]{\textwidth}
		\begin{figure}
			\centering
			\includegraphics[width=\textwidth]{./img/dime-gitflow-network-1-3.png}
		\end{figure}
	\end{minipage}

	\vspace{-.5cm}
	\begin{minipage}[t][5cm][t]{\textwidth}
		\begin{itemize}
			\setlength\itemsep{.5em}
			\item Merge the PR and delete the \textit{feature} branch
			\item The green box represents
			a full \textit{branch-PR-merge} cycle in the network graph
		\end{itemize}
	\end{minipage}

\end{frame}

\begin{frame}
	\frametitle{Nested branch-PR-merge cycles}

	\huge\centering \textbf{Nested branch-PR-merge cycles}

\end{frame}

\begin{frame}
	\frametitle{Nested branch-PR-merge cycles}
	\begin{columns}[c]

		\column{.65\textwidth} % Left column and width
		\begin{itemize}
			\setlength\itemsep{.5em}
			\item Many high-level tasks are not simple
			- multiple team members working over multiple months
			\item Cleaning the baseline data,
			for example, is one big task composed of many smaller tasks
			\item Solve this by using nested \textit{branch-PR-merge} cycles
			\item Third type of branch: \textbf{Develop branch}
			- a branch that is a high-level task that
			will have several \textit{feature} branches
		\end{itemize}

		\column{.35\textwidth} % Right column and width

		\begin{figure}
			\centering
			\includegraphics[width=.65\textwidth]{./img/team-challenge.png}
		\end{figure}
	\end{columns}
\end{frame}

\begin{frame}
	\frametitle{Create a develop branch}

	\vspace{-.5cm}
	\begin{minipage}[t][5cm][t]{\textwidth}
		\begin{figure}
			\centering
			\includegraphics[width=\textwidth]{./img/dime-gitflow-network-2-1.png}
		\end{figure}
	\end{minipage}

	\vspace{-.5cm}
	\begin{minipage}[t][5cm][t]{\textwidth}
		\begin{itemize}
			\setlength\itemsep{.5em}
			\item Create a \textit{develop} branch
			- name it after the high-level task
			- complete set up stage
			\item \textit{Develop} branches often do not have a Github issue
			- but the \textit{feature} branches off of it
			often have corresponding GitHub issues
			\item Include \textit{develop} branch name as prefix
			to all \textit{feature} branches off it
		\end{itemize}
	\end{minipage}
\end{frame}

\begin{frame}
	\frametitle{Create feature branches of the develop branch}

	\vspace{-.5cm}
	\begin{minipage}[t][5cm][t]{\textwidth}
		\begin{figure}
			\centering
			\includegraphics[width=\textwidth]{./img/dime-gitflow-network-2-2.png}
		\end{figure}
	\end{minipage}

	\vspace{-.5cm}
	\begin{minipage}[t][5cm][t]{\textwidth}
		\begin{itemize}
			\setlength\itemsep{.5em}
			\item Create as many \textit{feature} branches off
			the \textit{develop} branch as you need
			\item Split up the high-level task in as many smaller tasks as needed
			- try to split up tasks until each is small enough
			for one person to complete in max a week or two
			\item Create new \textit{feature} branches off
			the \textit{develop} branch even after it progresses
		\end{itemize}
	\end{minipage}
\end{frame}

\begin{frame}
	\frametitle{Work in the branches}

	\vspace{-.5cm}
	\begin{minipage}[t][5cm][t]{\textwidth}
		\begin{figure}
			\centering
			\includegraphics[width=\textwidth]{./img/dime-gitflow-network-2-3.png}
		\end{figure}
	\end{minipage}

	\vspace{-.5cm}
	\begin{minipage}[t][5cm][t]{\textwidth}
		\begin{itemize}
			\setlength\itemsep{.5em}
			\item Keep working on the \textit{feature} branches until
			they are done
			\item When a task is complete: assign a reviewer
			- make comments/suggestions - accept/reject/discuss
		\end{itemize}
	\end{minipage}
\end{frame}

\begin{frame}
	\frametitle{Merge a feature branch}

	\vspace{-.5cm}
	\begin{minipage}[t][5cm][t]{\textwidth}
		\begin{figure}
			\centering
			\includegraphics[width=\textwidth]{./img/dime-gitflow-network-2-4.png}
		\end{figure}
	\end{minipage}

	\vspace{-.5cm}
	\begin{minipage}[t][5cm][t]{\textwidth}
		\begin{itemize}
			\setlength\itemsep{.5em}
			\item The branch \texttt{baseline-clean-exp} was
			approved in the last step in the work stage and
			was merged and deleted in the merge stage
			\item The orange box is one \textit{branch-PR-merge} cycle
		\end{itemize}
	\end{minipage}
\end{frame}

\begin{frame}
	\frametitle{Merge a feature branch}

	\vspace{-.5cm}
	\begin{minipage}[t][5cm][t]{\textwidth}
		\begin{figure}
			\centering
			\includegraphics[width=\textwidth]{./img/dime-gitflow-network-2-5.png}
		\end{figure}
	\end{minipage}

	\vspace{-.5cm}
	\begin{minipage}[t][5cm][t]{\textwidth}
		\begin{itemize}
			\setlength\itemsep{.5em}
			\item The branch \texttt{baseline-clean-hh} was then
			also approved in the last step in the work stage and
			was merged and deleted in the merge stage
			\item The blue box is also a \textit{branch-PR-merge} cycle
		\end{itemize}
	\end{minipage}
\end{frame}

\begin{frame}
	\frametitle{Merge a develop branch}

	\vspace{-.5cm}
	\begin{minipage}[t][5cm][t]{\textwidth}
		\begin{figure}
			\centering
			\includegraphics[width=\textwidth]{./img/dime-gitflow-network-2-6.png}
		\end{figure}
	\end{minipage}

	\vspace{-.5cm}
	\begin{minipage}[t][5cm][t]{\textwidth}
		\begin{itemize}
			\setlength\itemsep{.5em}
			\item Then the branch \texttt{baseline} is ready for a review
			- assign a reviewer - comment/suggest - accept/reject/discuss
			\item The green box is also a \textit{branch-PR-merge} cycle
			\item This network graph shows
			a nested \textit{branch-PR-merge} cycle
		\end{itemize}
	\end{minipage}
\end{frame}


\begin{frame}
	\frametitle{Work directly in main/develop}

	\vspace{-.5cm}
	\begin{minipage}[t][5cm][t]{\textwidth}
		\begin{figure}
			\centering
			\includegraphics[width=\textwidth]{./img/dime-gitflow-network-workdirectly.png}
		\end{figure}
	\end{minipage}

	\vspace{-.5cm}
	\begin{minipage}[t][5cm][t]{\textwidth}
		\begin{itemize}
			\setlength\itemsep{.5em}
			\item <1->When could it be ok to work
			directly on a \textit{main}/\textit{develop} branch?
			- see red arrows
			\item <2->When updating documentation!
			\begin{itemize}
				\setlength\itemsep{.5em}
				\item <2->Documentation about the repo
				in the \textit{main} branch
				\item <2->Documentation about the high-level task
				in the \textit{develop} branch
			\end{itemize}
		\end{itemize}
	\end{minipage}
\end{frame}

\begin{frame}
	\frametitle{The network graph}

	\vspace{-.5cm}
	\begin{minipage}[t][5cm][t]{\textwidth}
		\begin{figure}
			\centering
			\includegraphics[width=\textwidth]{./img/dime-gitflow-network-names.png}
		\end{figure}
	\end{minipage}

	\vspace{-1cm}
	\begin{minipage}[t][5cm][t]{\textwidth}
		\begin{itemize}
			\setlength\itemsep{.4em}
			\item There is no technical git difference between
			the Gitflow types of branches
			- the only difference is what workflow that is expected
			\begin{itemize}
				\setlength\itemsep{.5em}
				\item \textbf{Main branch}:
				The branch where all branches originates from
				\item \textbf{Develop branch}:
				Non-\textit{main} branch with \textit{feature} branches
				\item \textbf{Feature branch}:
				Branches of the \textit{main} branch or
				\textit{develop} branches with no other branches from it
			\end{itemize}
		\end{itemize}
	\end{minipage}
\end{frame}


\begin{frame}
	\frametitle{Nest cycles in the work stage}
	\begin{columns}[c]

		\column{.35\textwidth} % Left column and width
		\begin{itemize}
			\setlength\itemsep{1em}
			\item \textit{Branch-PR-merge} cycles should be nested
			in the work stage
			\item The \textit{develop} branch is kept
			in the \textit{Commit edits} step while
			the \textit{branch-PR-merge} cycles are completed
			for the feature branches
			\item If needed, convert a \textit{feature} branch
			to a \textit{develop} branch here
		\end{itemize}

		\column{.65\textwidth} % Right column and width
		\vspace{-.75cm}
		\begin{figure}
			\centering
			\includegraphics[width=\textwidth]{./img/branch-pr-merge-cycle-S2-1.png}
		\end{figure}
	\end{columns}
\end{frame}

\begin{frame}
	\frametitle{Thank you!}
	\huge\centering \textbf{Questions?}

	\vspace{1cm}
	\normalsize Slides can be found here: \url{https://osf.io/e54gy/}
\end{frame}

\begin{frame}{Links}
	Useful links:	
	\begin{itemize}
		\item DIME Analytics GitHub Templates (for example .gitignore): \trainingURL{https://github.com/worldbank/DIMEwiki/tree/master/Topics/GitHub}
		\item DIME Analytics GitHub Roles: \trainingURL{https://github.com/worldbank/dime-github-trainings/blob/master/GitHub-resources/DIME-GitHub-Roles/DIME-GitHub-roles.md}
		\item Markdown cheat sheet (how to format text on GitHub.com):  \trainingURL{https://www.markdownguide.org/cheat-sheet/}
	\end{itemize}
\end{frame}




\begin{frame}{Version control}
	At the point of compiling this Beamer presentation, the most recent commit was:

	\commiturl

\end{frame}


\section{Appendix}

\begin{frame}{Create branch on GitHub.com}
\label{new-branch}

\begin{figure}
	\centering
	\includegraphics[width=.9\textwidth]{./img/new-branch-1.png}
\end{figure}
\end{frame}

\begin{frame}{Create branch on GitHub.com}
\begin{figure}
	\centering
	\includegraphics[width=.9\textwidth]{./img/new-branch-2.png}
\end{figure}
\end{frame}

\begin{frame}{Create branch on GitHub.com}
\begin{figure}
	\centering
	\includegraphics[width=.9\textwidth]{./img/new-branch-3.png}
\end{figure}
\hyperlink{Create a branch}{\beamerreturnbutton{contents page}}
\end{frame}

\end{document}
