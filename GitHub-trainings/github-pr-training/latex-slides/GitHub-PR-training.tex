% ---------------------------- Preamble starts here ----------------------------

\documentclass[aspectratio=169]{beamer} %Remove [aspectratio=169] to get non-wide 4:3 slide aspect ratio

%-----------------------------------------------
% --- Set beamer theme
\usetheme{Metropolis}
\setbeamertemplate{footline}{}				% Remove automatic footer
\setbeamertemplate{navigation symbols}{}	% Comment this line to display navigation symbols

%-----------------------------------------------
% Load i2i symbol
\addtobeamertemplate{frametitle}{}{%
\begin{textblock*}{\linewidth}(0cm,7.4cm) % Replace with (0cm, 8cm) if using non-wide slide aspect
	\includegraphics[width=\linewidth]{../../Common-Resources/img/Footer.png}
\end{textblock*}}

\setbeamertemplate{footline}{\hfill\insertframenumber/\inserttotalframenumber}

%-----------------------------------------------
% --- Load packages
\usepackage{textpos}		% To align objects correctly
\usepackage{multicol}		% To right in multiple columns
\usepackage{color}			% To color text

%-----------------------------------------------
% --- Include link to last commit
\usepackage{xstring}
\usepackage{catchfile}

%Set this user input
\newcommand{\gitfolder}{../../../.git} %relative path to .git folder from .tex doc
\newcommand{\reponame}{worldbank/dime-github-trainings} % Name of account and repo be set in URL

%Based on this https://tex.stackexchange.com/questions/455396/how-to-include-the-current-git-commit-id-and-branch-in-my-document
\CatchFileDef{\headfull}{\gitfolder/HEAD.}{} 				%Get path to head file for checked out branch
\StrGobbleRight{\headfull}{1}[\head]						%Remove end of line character
\StrBehind[2]{\head}{/}[\branch]							%Parse out the path only
\CatchFileDef{\commit}{\gitfolder/refs/heads/\branch.}{}	%Get the content of the branch head
\StrGobbleRight{\commit}{1}[\commithash]					%Remove end of line characted

%Build the URL to this commit based on the information we now have
\newcommand{\commiturl}{\url{https://github.com/\reponame/commit/\commithash}}

%-----------------------------------------------
% --- Add your information here
\title{GitHub - Pull Request training}
\author{DIME Analytics}
\institute{DIME - The World Bank - \trainingURL{https://www.worldbank.org/en/research/dime}}
\date{\today}

\newcommand{\trainingURL}[1]{{\color{blue}\url{#1}}}

\newcommand{\traininerUsername}{dime-wb-trainings}
\newcommand{\repoName}{\traininerUsername/pr-training-dec21}
\newcommand{\trainingRepoURL}[1]{\trainingURL{github.com/\repoName #1}}
\newcommand{\trainerEmail}{\trainingURL{kbjarkefur@worldbank.org} }


% ---------------------------- Preamble ends here ----------------------------

\begin{document}

\begin{frame}
\includegraphics[width=\textwidth]{../../Common-Resources/img/Header.png}
\vspace{-0.2cm}
\titlepage 	 % Opening slide, prints inform
\end{frame}

\begin{frame}
	\frametitle{Prerequisites}
	\begin{columns}[c]

		\column{.65\textwidth} % Left column and width

		\large \textbf{Who is this training for?}

		\vspace{1em}

		Anyone comfortable branching, committing and merging in Git/GitHub
		who is ready to be introduced to a more advanced workflow
		using Git/GitHub to ensure high quality code.

		\column{.25\textwidth} % Left column and width

		\begin{minipage}[t][3.5cm][t]{\textwidth}
			\begin{figure}
				\centering
				\includegraphics[width=.8\textwidth]{./img/git-icon.png}
			\end{figure}
		\end{minipage}
		\vspace{-.5cm}
		\begin{minipage}[t][3.5cm][t]{\textwidth}
			\begin{figure}
				\centering
				\includegraphics[width=.8\textwidth]{./img/github-icon.png}
			\end{figure}
		\end{minipage}

		\column{.05\textwidth} % Left column and width

	\end{columns}
\end{frame}

\begin{frame}
	\frametitle{Content}

	\Large\centering \textbf{THIS TRAINING HAS THREE PARTS}

	\vspace{1.2em}

	\raggedright
	\begin{columns}[T]

		\column{.01\textwidth}
		\column{.3\textwidth} % Left column and width	
			\large \textbf{Part 1}
			
			\vspace{1em}
			\raggedright
			\normalfont A brief recap of using branches in Git

		\column{.01\textwidth}
		\column{.3\textwidth} % Left column and width
			\large \textbf{Part 2}
			
			\vspace{1em}
			\raggedright
			\normalfont Introducing the \textit{branch-PR-merge} cycle for how to use the review features in a PR
			
		\column{.01\textwidth}
		\column{.3\textwidth} % Left column and width
			\large \textbf{Part 3}
			
			\vspace{1em}
			\raggedright
			\normalfont Gitflow - a philosophy for how to work with branches and merge/PRs in Git
		
		\column{.01\textwidth}

	\end{columns}
\end{frame}


\begin{frame}
	\frametitle{The branch-PR-merge cycle}
	
	\begin{columns}[c]
		
		\column{.6\textwidth} % Right column and width
		\vspace{-1cm}
		\begin{figure}
			\centering
			\includegraphics[width=\textwidth]{./img/training-repo.png}
		\end{figure}
	
		\column{.4\textwidth} % Right column and width
		All screenshots in this slide comes from this repo: \trainingRepoURL{}
		
	\end{columns}
\end{frame}

\section{Part 1: \newline A brief recap to branches}

\begin{frame}
	\frametitle{Brief intro to branches}
	\begin{columns}[c]
		
		\column{.68\textwidth} % Left column and width
		
		Branches is the killer feature of Git. 
		It allows us to approach tasks in a non-linear way,
		which is similar to how human brains
		and especially teams of humans work.
		\vspace{1.5em}	

		Thanks to branches we can - without creating conflicts -
		test multiple solutions to as task at the same time, or 
		have multiple people working at different tasks at the same time.
		\vspace{1.5em}
			
		Branches are often visualized in network graphs.
		
		\column{.25\textwidth} % Left column and width
		
		\begin{minipage}[t][6.5cm][t]{\textwidth}
			\begin{figure}
				\centering
				\includegraphics[width=.55\textwidth]{./img/pr-branch.png}
			\end{figure}
		\end{minipage}
		
		\column{.05\textwidth} % Left column and width
		
	\end{columns}
\end{frame}


\begin{frame}
	\frametitle{What are pull request?}
	\begin{columns}[c]
		
		\column{.68\textwidth} % Left column and width
		
		\textit{Pull requests} (PRs) are part of GitHub, but not part of Git.	
		\vspace{2em}
		
		GitHub is to Git, what Gmail is to email.
		\vspace{2em}
		
		A PR is a request to merge one branch into another,
		and adds great QA and review features before the merge.
		\column{.25\textwidth} % Left column and width
		
		\begin{minipage}[t][6.5cm][t]{\textwidth}
			\begin{figure}
				\centering
				\includegraphics[width=.55\textwidth]{./img/pr-highlight.png}
			\end{figure}
		\end{minipage}
		
		\column{.05\textwidth} % Left column and width
		
	\end{columns}
\end{frame}

\section{Part 2: \newline The branch-PR-merge cycle}

\begin{frame}
\frametitle{The branch-PR-merge cycle}

	\begin{columns}[c]

		\column{.4\textwidth} % Left column and width
		\large Complete \textbf{every} data work task in your project with a \textit{branch-PR-merge} cycle
		\vspace{.5cm}\newline
		\large The \textit{branch-PR-merge} cycle consists of three stages,
		we will focus on the \textit{work stage} today
		\vspace{.5cm}\newline
		\large A more detailed version in Appendix A of these slides.

		\column{.6\textwidth} % Right column and width
		\vspace{-.75cm}
		\begin{figure}
			\centering
			\includegraphics[width=\textwidth]{./img/branch-pr-merge-cycle.png}
		\end{figure}

	\end{columns}
\end{frame}


\begin{frame}
	\frametitle{Set-up Stage}
	\begin{columns}[c]

		\column{.4\textwidth} % Left column and width

		\Large \textbf{SET-UP STAGE:}
		\vspace{.8em}
	
		\normalsize
		Create a branch for this task!
		\vspace{.5cm}\newline
		Part 3 is all about when and how many branches to create
		\vspace{.5cm}\newline
		See Appendix B for step-by-step instructions on how to create branches
		
		\column{.6\textwidth} % Right column and width
		\vspace{-.75cm}
		\begin{figure}
			\centering
			\includegraphics[width=\textwidth]{./img/branch-pr-merge-cycle-S1a.png}
		\end{figure}

	\end{columns}
\end{frame}


\begin{frame}
	\frametitle{Work Stage}
	\begin{columns}[c]
		
		\column{.4\textwidth} % Left column and width
		
		\Large \textbf{WORK STAGE:}
		\vspace{.8em}
		
		\normalsize
		Most time in a branch-pr-merge cycle is spent in the work stage.
		\vspace{.5cm}\newline
		Research teams typically under-utilize features 
		in the \textit{Review} and \textit{Discuss} steps, 
		despite this is the real added value of PRs.\vspace{.5cm}\newline
		The work stage should be repeated as many times as needed.
		
		\column{.6\textwidth} % Right column and width
		\vspace{-.75cm}
		\begin{figure}
			\centering
			\includegraphics[width=\textwidth]{./img/branch-pr-merge-cycle-S2.png}
		\end{figure}
		
	\end{columns}
\end{frame}

\begin{frame}
	\frametitle{Work Stage : Commit edits}
	\begin{columns}[c]
		
		\column{.4\textwidth} % Left column and width
		
		\normalsize
		In the first step team member working on this task
		commits edits as usual to the new branch 
		\vspace{.5cm}\newline
		The PR needs to be opened before moving to the \textit{Review PR} step
		\vspace{.5cm}\newline
		We will call the person making the commits the \textit{Author} and the person reviewing the PR the \textit{Reviewer}. This can be two people, groups of people or one person reviewing their own work.
		
		\column{.6\textwidth} % Right column and width
		\vspace{-.75cm}
		\begin{figure}
			\centering
			\includegraphics[width=\textwidth]{./img/branch-pr-merge-cycle-S2a.png}
		\end{figure}
		
	\end{columns}
\end{frame}

\begin{frame}
	\frametitle{Create the PR}
	\begin{columns}[c]
		
		\column{.50\textwidth} % Right column and width
		\vspace{-.75cm}
		\begin{figure}
			\centering
			\includegraphics[width=\textwidth]{./img/create-pr-0.png}
		\end{figure}
		
		\column{.50\textwidth} % Left column and width
		\begin{itemize}
			\setlength\itemsep{1em}
			\item Typically done by the Author		
			\item Select the two branches you want to merge
			\item Remember a merge is not symmetric 
			- only edits in the \textit{compare} branch are merged into to \textit{base} branch.
			\item Check the tabs in the \textit{Commits}, \textit{Files changed} etc. bar
			to make sure you are creating a PR for the right branches - merges are often messy to revert
		\end{itemize}
		
	\end{columns}
\end{frame}

\begin{frame}
	\frametitle{Document the PR}
	\begin{columns}[c]
		
		\column{.50\textwidth} % Right column and width
		\vspace{-.25cm}
		\begin{figure}
			\centering
			\includegraphics[width=\textwidth]{./img/create-pr-1.png}
		\end{figure}
		
		\column{.50\textwidth} % Left column and width
		\begin{itemize}
			\setlength\itemsep{1.5em}
			\item Merged PRs are milestones that can be browsed in the future - 
			document these milestones
			\item Why did the Author work on this task?
			\item Making people reviewers and assignees as well as 
			using labels and projects are neat features, but not necessary 
		\end{itemize}
		
		\vspace{.25cm}
		\footnotesize See PR at \trainingRepoURL{/pull/1}
		
	\end{columns}
\end{frame}


\begin{frame}
	\frametitle{Work Stage : Review PR}
	\begin{columns}[c]
		
		\column{.4\textwidth} % Left column and width
		
		\normalsize
		Review a PR when either: \vspace{.3cm}
		\begin{itemize}
			\item the Author is completed 
			\item the Author wants help
			\item after a set amount of time 
		\end{itemize}
	
		\vspace{.5cm}

		The PR can be opened as a \textit{Draft PR} already after one commit. 
		More on this in Appendix A.

		
		\column{.6\textwidth} % Right column and width
		\vspace{-.75cm}
		\begin{figure}
			\centering
			\includegraphics[width=\textwidth]{./img/branch-pr-merge-cycle-S2b.png}
		\end{figure}
		
	\end{columns}
\end{frame}

\begin{frame}
	\frametitle{Overall comments}
	\begin{columns}[c]
		
		\column{.6\textwidth} % Right column and width
		\vspace{-.4cm}
		\begin{figure}
			\centering
			\includegraphics[width=.9\textwidth]{./img/pr-comment.png}
		\end{figure}
		
		\column{.4\textwidth} % Left column and width
		\begin{itemize}
			\setlength\itemsep{1em}
			\item If applicable/possible, always start by running the code
			\item Provide overall comments in the \textit{``Conversation"} tab. Code does or does not run, not possible to follow the code etc.
			\item Tag the Author or Reviewer your comment is meant for
		\end{itemize}
		
	\end{columns}
\end{frame}

\begin{frame}
	\frametitle{Review the PR}
	\begin{columns}[c]

		\column{.40\textwidth} % Right column and width
		\vspace{-.2cm}
		\begin{figure}
			\centering
			\includegraphics[width=.6\textwidth]{./img/review-1.png}
		\end{figure}

		\column{.60\textwidth} % Left column and width
		\begin{itemize}
			\setlength\itemsep{1.5em}
			\item Next, go to the \textit{``Files changed''} tab
			- this is the summary of all changes in all commits in this PR
			\item In this tab we make comments to specific lines of code which is always much more useful than overall comments
		\end{itemize}

	\end{columns}
\end{frame}

\begin{frame}
	\frametitle{Line comments}
	\begin{columns}[c]
		
		\column{.55\textwidth} % Right column and width
		\vspace{-.4cm}
		\begin{figure}
			\centering
			\includegraphics[width=.9\textwidth]{./img/line-comment-1.png}
		\end{figure}
		\vspace{-.3cm}
		\begin{figure}
			\centering
			\includegraphics[width=.9\textwidth]{./img/line-comment-2.png}
		\end{figure}
		
		\column{.45\textwidth} % Left column and width
		\begin{itemize}
			\setlength\itemsep{1em}
			\item You can start a thread anywhere in the code in this tab
			\item \textit{Comment one line}: Hold with the mouse over the line number 
			for a line you want to comment on and click the plus sign - orange circle
			\item \textit{Comment multiple lines:} Press shift while 
			click-and-drag mouse over multiple lines - black circle
		\end{itemize}
		
	\end{columns}
\end{frame}

\begin{frame}
	\frametitle{Line suggestions}
	\begin{columns}[c]

		\column{.50\textwidth} % Right column and width
		\vspace{-.6cm}
		\begin{figure}
			\centering
			\includegraphics[width=\textwidth]{./img/suggestion-1.png}
		\end{figure}
		\vspace{-.3cm}
		\begin{figure}
			\centering
			\includegraphics[width=\textwidth]{./img/suggestion-2.png}
		\end{figure}

		\column{.50\textwidth} % Left column and width
		\begin{itemize}
			\setlength\itemsep{.74em}
			\item You can suggest edits to one or multiple lines of code 
			directly in this comment field
			
			\item Click the suggest code button - red circle - 
			and edit the code that appears
			
			\item The Author will then see the exact suggestion, 
			and can commit it with the click of a button
			
			\item Works really well when tweaking the code in the PR,
			but only works well for smaller edits
		\end{itemize}

	\end{columns}
\end{frame}

\begin{frame}
	\frametitle{Discuss and accept/reject}
	\begin{columns}[c]

		\column{.35\textwidth} % Left column and width
		\begin{itemize}
			\setlength\itemsep{1em}
			\item The Author discusses, accepts or rejects suggestions
			and replies to comments
			\item The Author, the Reviewer(s)
			and/or anyone else in the project team then
			decide if the work stage needs to be repeated
		\end{itemize}

		\column{.6\textwidth} % Right column and width
		\vspace{-.75cm}
		\begin{figure}
			\centering
			\includegraphics[width=\textwidth]{./img/branch-pr-merge-cycle-S2c.png}
		\end{figure}

	\end{columns}
\end{frame}

\begin{frame}
	\frametitle{Conversation summary}
	\begin{columns}[c]
		
		\column{.4\textwidth} % Right column and width
		\vspace{-.2cm}
		\begin{figure}
			\centering
			\includegraphics[width=\textwidth]{./img/conversation-summary.png}
		\end{figure}
		
		\column{.60\textwidth} % Left column and width
		\begin{itemize}
			\setlength\itemsep{.7em}
			\item Back in the \textit{``Conversation"} tab, all comments and suggestions are summarized. 
			\item Resolve comments by clicking the \textit{``Resolve conversation"} button. \item Resolve suggestions by accepting them with the \textit{``Commit Suggestion"} button or ignore the suggestion by clicking \textit{``Resolve conversation"}.
			\item Keep reviewing the PR until all these comments are resolved.
			\item If applicable/possible, always test run the code a final time
		\end{itemize}
		
	\end{columns}
\end{frame}


\begin{frame}
	\frametitle{Branch-PR-Merge }
	\begin{columns}[c]

		\column{.35\textwidth} % Right column and width
		\begin{figure}
			\centering
			\includegraphics[width=.6\textwidth]{./img/qa.png}
		\end{figure}

		\column{.55\textwidth} % Left column and width
		\begin{itemize}
			\setlength\itemsep{1em}
			\item When all suggestions and comments
			have been responded to,
			the team should discuss if a task either 
			is done or need more work 
			\item Repeat the work stage 
			if significant work is still required
			\item Move on to the merge stage 
			only when the task is complete and 
			the team agrees the code is high-quality
		\end{itemize}

	\end{columns}
\end{frame}


\begin{frame}
	\frametitle{Merge Stage}
	\begin{columns}[c]
		
		\column{.4\textwidth} % Left column and width
		
		\Large \textbf{MERGE STAGE:}
		\vspace{.8em}
		
		\normalsize
		Last stage of the \textit{branch-pr-merge} cycle is to merge the PR
		once everyone have agreed to the suggested changes
		\vspace{.5cm}\newline
		Always delete the old branch - recreate that branch from the \textit{base} branch
		if you want to keep working in a branch with the same name.
		
		\column{.6\textwidth} % Right column and width
		\vspace{-.75cm}
		\begin{figure}
			\centering
			\includegraphics[width=\textwidth]{./img/branch-pr-merge-cycle-S3.png}
		\end{figure}
		
	\end{columns}
\end{frame}

\section{Part 3: \newline Gitflow - When, how and how much to branch}

\begin{frame}
	\frametitle{The network graph}

	\vspace{-.5cm}
	\begin{minipage}[t][5cm][t]{\textwidth}
		\begin{figure}
			\centering
			\includegraphics[width=\textwidth]{./img/dime-gitflow-network.png}
		\end{figure}
	\end{minipage}

	\vspace{-.5cm}
	\begin{minipage}[t][5cm][t]{\textwidth}
		\begin{itemize}
			\setlength\itemsep{.5em}
			\item A typical git network graph - each dot is a commit
			\item Several branches have been opened and merged,
			but currently only the \textit{main} branch is open
		\end{itemize}
	\end{minipage}

\end{frame}

\begin{frame}
	\frametitle{Gitflow}
	\begin{columns}[c]

		\column{.65\textwidth} % Left column and width
		\begin{itemize}
			\setlength\itemsep{.5em}
			\item \textbf{Gitflow} is an idea or a philosophy 
			of how to organize work in git - it is not a software you install
			\item It is developed by computer scientists and
			you will find a lot of resources for it online
			\item We will present a simplified version adapted for research
			\item Two branch types (later we add a third type):
			\begin{itemize}
				\item The \textbf{main branch} (formerly master branch)
				- should never be worked in directly
				\item \textbf{Feature branches} - this is where all work should be done
			\end{itemize}
		\end{itemize}

		\column{.35\textwidth} % Right column and width
		\vspace{-.75cm}
		\begin{figure}
			\centering
			\includegraphics[width=.75\textwidth]{./img/organization.png}
		\end{figure}
	\end{columns}
\end{frame}



\begin{frame}
	\frametitle{Simple branch-PR-merge cycle}

	\huge\centering \textbf{Simple branch-PR-merge cycle}

\end{frame}


\begin{frame}
	\frametitle{Start a branch-PR-merge cycle}

	\vspace{-.5cm}
	\begin{minipage}[t][5cm][t]{\textwidth}
		\begin{figure}
			\centering
			\includegraphics[width=\textwidth]{./img/dime-gitflow-network-0.png}
		\end{figure}
	\end{minipage}

	\vspace{-.5cm}
	\begin{minipage}[t][5cm][t]{\textwidth}
		\begin{itemize}
			\setlength\itemsep{.5em}
			\item This can either be the beginning of a repo
			or at a point where all previous branches were merged
			to the main branch
			\item Let's say your next task is to set up folders
			- apply the \textit{branch-PR-merge} cycle
		\end{itemize}
	\end{minipage}

\end{frame}


\begin{frame}
	\frametitle{Set up stage - 1}

	\vspace{-.5cm}
	\begin{minipage}[t][5cm][t]{\textwidth}
		\begin{figure}
			\centering
			\includegraphics[width=\textwidth]{./img/dime-gitflow-network-1-0.png}
		\end{figure}
	\end{minipage}

	\vspace{-.5cm}
	\begin{minipage}[t][5cm][t]{\textwidth}
		\begin{itemize}
			\setlength\itemsep{.5em}
			\item Create a new \textit{feature} branch at the point
			where you want the the \textit{branch-PR-merge} cycle
			to start from
			\item Two branches pointing to the same commit
			are identical by definition
		\end{itemize}
	\end{minipage}

\end{frame}


\begin{frame}
	\frametitle{Set up stage - 2}

	\vspace{-.5cm}
	\begin{minipage}[t][5cm][t]{\textwidth}
		\begin{figure}
			\centering
			\includegraphics[width=\textwidth]{./img/dime-gitflow-network-1-1.png}
		\end{figure}
	\end{minipage}

	\vspace{-.5cm}
	\begin{minipage}[t][5cm][t]{\textwidth}
		\begin{itemize}
			\setlength\itemsep{.5em}
			\item The Author starts working on the task in the new branch
			\item As soon as one commit is made you can open a new PR 
			where this task can be reviewed and discussed - great case for a draft PR
		\end{itemize}
	\end{minipage}

\end{frame}

\begin{frame}
	\frametitle{Work stage}

	\vspace{-.5cm}
	\begin{minipage}[t][5cm][t]{\textwidth}
		\begin{figure}
			\centering
			\includegraphics[width=\textwidth]{./img/dime-gitflow-network-1-2.png}
		\end{figure}
	\end{minipage}

	\vspace{-.5cm}
	\begin{minipage}[t][5cm][t]{\textwidth}
		\begin{itemize}
			\setlength\itemsep{.5em}
			\item Complete the task using one or many commits
			- in this example only one more commit was needed
			\item Assign someone to review the PR
			- iterate the work stage of the branch-PR-merge cycle until everyone is satisfied
		\end{itemize}
	\end{minipage}

\end{frame}

\begin{frame}
	\frametitle{Merge stage}

	\vspace{-.5cm}
	\begin{minipage}[t][5cm][t]{\textwidth}
		\begin{figure}
			\centering
			\includegraphics[width=\textwidth]{./img/dime-gitflow-network-1-3.png}
		\end{figure}
	\end{minipage}

	\vspace{-.5cm}
	\begin{minipage}[t][5cm][t]{\textwidth}
		\begin{itemize}
			\setlength\itemsep{.5em}
			\item Merge the PR and delete the \textit{feature} branch
			\item The green box represents
			a full \textit{branch-PR-merge} cycle in the network graph
		\end{itemize}
	\end{minipage}

\end{frame}

\begin{frame}
	\frametitle{Nested branch-PR-merge cycles}

	\huge\centering \textbf{Nested branch-PR-merge cycles}

\end{frame}

\begin{frame}
	\frametitle{Nested branch-PR-merge cycles}
	\begin{columns}[c]

		\column{.65\textwidth} % Left column and width
		\begin{itemize}
			\setlength\itemsep{.5em}
			\item Many high-level tasks are not simple
			- multiple team members working over multiple months
			\item Cleaning the baseline data,
			for example, is one big task composed of many smaller tasks
			\item Solve this by using nested \textit{branch-PR-merge} cycles
			\item Third type of branch: \textbf{Develop branch}
			- a branch that is a high-level task that
			will have several \textit{feature} branches
		\end{itemize}

		\column{.35\textwidth} % Right column and width

		\begin{figure}
			\centering
			\includegraphics[width=.65\textwidth]{./img/team-challenge.png}
		\end{figure}
	\end{columns}
\end{frame}

\begin{frame}
	\frametitle{Create a develop branch}

	\vspace{-.5cm}
	\begin{minipage}[t][5cm][t]{\textwidth}
		\begin{figure}
			\centering
			\includegraphics[width=\textwidth]{./img/dime-gitflow-network-2-1.png}
		\end{figure}
	\end{minipage}

	\vspace{-.5cm}
	\begin{minipage}[t][5cm][t]{\textwidth}
		\begin{itemize}
			\setlength\itemsep{1em}
			\item Create a \textit{develop} branch
			- name it after the high-level task
			\item Include the name of the \textit{develop} branch as prefix
			to all \textit{feature} branches off it
		\end{itemize}
	\end{minipage}
\end{frame}

\begin{frame}
	\frametitle{Create feature branches of the develop branch}

	\vspace{-.5cm}
	\begin{minipage}[t][5cm][t]{\textwidth}
		\begin{figure}
			\centering
			\includegraphics[width=\textwidth]{./img/dime-gitflow-network-2-2.png}
		\end{figure}
	\end{minipage}

	\vspace{-.5cm}
	\begin{minipage}[t][5cm][t]{\textwidth}
		\begin{itemize}
			\setlength\itemsep{.5em}
			\item Create as many \textit{feature} branches off
			the \textit{develop} branch as you need
			\item Split up the high-level task in as many smaller tasks as needed
			- try to split up tasks until each is small enough
			for one person to complete in max a week or two
			\item Create new \textit{feature} branches off
			the \textit{develop} branch even after it progresses
		\end{itemize}
	\end{minipage}
\end{frame}

\begin{frame}
	\frametitle{Work in the branches}

	\vspace{-.5cm}
	\begin{minipage}[t][5cm][t]{\textwidth}
		\begin{figure}
			\centering
			\includegraphics[width=\textwidth]{./img/dime-gitflow-network-2-3.png}
		\end{figure}
	\end{minipage}

	\vspace{-.5cm}
	\begin{minipage}[t][5cm][t]{\textwidth}
		\begin{itemize}
			\setlength\itemsep{.5em}
			\item The Author keeps working on the \textit{feature} branches until
			the task is done
			\item When a task is complete: assign a reviewer
			- make comments/suggestions - accept/reject/discuss
		\end{itemize}
	\end{minipage}
\end{frame}

\begin{frame}
	\frametitle{Merge a feature branch}

	\vspace{-.5cm}
	\begin{minipage}[t][5cm][t]{\textwidth}
		\begin{figure}
			\centering
			\includegraphics[width=\textwidth]{./img/dime-gitflow-network-2-4.png}
		\end{figure}
	\end{minipage}

	\vspace{-.5cm}
	\begin{minipage}[t][5cm][t]{\textwidth}
		\begin{itemize}
			\setlength\itemsep{.5em}
			\item The branch \texttt{baseline-clean-exp} was
			approved in the last step in the work stage and
			was merged and deleted in the merge stage
			\item The orange box is one \textit{branch-PR-merge} cycle
		\end{itemize}
	\end{minipage}
\end{frame}

\begin{frame}
	\frametitle{Merge a feature branch}

	\vspace{-.5cm}
	\begin{minipage}[t][5cm][t]{\textwidth}
		\begin{figure}
			\centering
			\includegraphics[width=\textwidth]{./img/dime-gitflow-network-2-5.png}
		\end{figure}
	\end{minipage}

	\vspace{-.5cm}
	\begin{minipage}[t][5cm][t]{\textwidth}
		\begin{itemize}
			\setlength\itemsep{.5em}
			\item The branch \texttt{baseline-clean-hh} was then
			also approved in the last step in the work stage and
			was merged and deleted in the merge stage
			\item The blue box is also a \textit{branch-PR-merge} cycle
		\end{itemize}
	\end{minipage}
\end{frame}

\begin{frame}
	\frametitle{Merge a develop branch}

	\vspace{-.5cm}
	\begin{minipage}[t][5cm][t]{\textwidth}
		\begin{figure}
			\centering
			\includegraphics[width=\textwidth]{./img/dime-gitflow-network-2-6.png}
		\end{figure}
	\end{minipage}

	\vspace{-.5cm}
	\begin{minipage}[t][5cm][t]{\textwidth}
		\begin{itemize}
			\setlength\itemsep{.5em}
			\item Then the branch \texttt{baseline} is ready for a review
			- assign a reviewer - comment/suggest - accept/reject/discuss
			\item The green box is also a \textit{branch-PR-merge} cycle
			\item This network graph shows
			a nested \textit{branch-PR-merge} cycle
		\end{itemize}
	\end{minipage}
\end{frame}

\begin{frame}
	\frametitle{Nest cycles in the work stage}
	\begin{columns}[c]
		
		\column{.4\textwidth} % Left column and width
		\begin{itemize}
			\setlength\itemsep{1em}
			\item \textit{Branch-PR-merge} cycles should be nested
			in the work stage
			\item The \textit{develop} branch is kept
			in the \textit{Commit edits} step while
			the \textit{branch-PR-merge} cycles are completed
			for the feature branches
			\item If needed, convert a \textit{feature} branch
			to a \textit{develop} branch here
		\end{itemize}
		
		\column{.6\textwidth} % Right column and width
		\vspace{-.75cm}
		\begin{figure}
			\centering
			\includegraphics[width=\textwidth]{./img/nested-branch-pr-merge-cycle.png}
		\end{figure}
	\end{columns}
\end{frame}

\begin{frame}
	\frametitle{Work directly in main/develop}

	\vspace{-.5cm}
	\begin{minipage}[t][5cm][t]{\textwidth}
		\begin{figure}
			\centering
			\includegraphics[width=\textwidth]{./img/dime-gitflow-network-workdirectly.png}
		\end{figure}
	\end{minipage}

	\vspace{-.5cm}
	\begin{minipage}[t][5cm][t]{\textwidth}
		\begin{itemize}
			\setlength\itemsep{.5em}
			\item <1->When could it be ok to work
			directly on a \textit{main}/\textit{develop} branch?
			- see red arrows
			\item <2->When updating documentation!
			\begin{itemize}
				\setlength\itemsep{.5em}
				\item <2->Documentation about the repo
				in the \textit{main} branch
				\item <2->Documentation about the high-level task
				in the \textit{develop} branch
			\end{itemize}
		\end{itemize}
	\end{minipage}
\end{frame}

\begin{frame}
	\frametitle{The network graph}

	\vspace{-.5cm}
	\begin{minipage}[t][5cm][t]{\textwidth}
		\begin{figure}
			\centering
			\includegraphics[width=\textwidth]{./img/dime-gitflow-network-names.png}
		\end{figure}
	\end{minipage}

	\vspace{-1cm}
	\begin{minipage}[t][5cm][t]{\textwidth}
		\begin{itemize}
			\setlength\itemsep{.4em}
			\item There is no difference to git between
			different Gitflow types of branches
			- the only difference is how team members use them
			\begin{itemize}
				\setlength\itemsep{.5em}
				\item \textbf{Main branch}:
				The branch where all branches originates from
				\item \textbf{Develop branch}:
				Non-\textit{main} branch with \textit{feature} branches
				\item \textbf{Feature branch}:
				Branches of the \textit{main} branch or
				\textit{develop} branches with no other branches from it
			\end{itemize}
		\end{itemize}
	\end{minipage}
\end{frame}

\begin{frame}
	\frametitle{Thank you!}
	\huge\centering \textbf{Questions?}

	\vspace{1cm}
	\normalsize Most up to date version of these slides here: \url{https://osf.io/e54gy}
	
	\vspace{.2cm}
	\normalsize The exact version of the slides used today here: \url{https://osf.io/ker4u}
\end{frame}

\begin{frame}{Links}
	Useful links:	
	\begin{itemize}
		\item DIME Analytics GitHub Templates (for example .gitignore): \trainingURL{https://github.com/worldbank/DIMEwiki/tree/master/Topics/GitHub}
		\item DIME Analytics GitHub Roles: \trainingURL{https://github.com/worldbank/dime-github-trainings/blob/master/GitHub-resources/DIME-GitHub-Roles/DIME-GitHub-roles.md}
		\item Markdown cheat sheet (how to format text on GitHub.com):  \trainingURL{https://www.markdownguide.org/cheat-sheet/}
	\end{itemize}
\end{frame}




\section{Appendix A : A more detailed version of the branch-PR-merge cycle }

\begin{frame}
	\frametitle{Nest cycles in the work stage}
	\begin{columns}[c]
		
		\column{.4\textwidth} % Left column and width
		\begin{itemize}
			\setlength\itemsep{1em}
			\item This is the full \textit{branch-PR-merge} cycle
			\item The difference is the issue and the draft-pr
			\item Issues allows a discussion even before the team decides to work on that task. 
			Discussions in PRs tend to be more focused on how a task is solved.
		\end{itemize}
		
		\column{.6\textwidth} % Right column and width
		\vspace{-.75cm}
		\begin{figure}
			\centering
			\includegraphics[width=\textwidth]{./img/full-branch-pr-merge-cycle.png}
		\end{figure}
	\end{columns}
\end{frame}


\section{Appendix B : How to create a branch}

\begin{frame}{Create branch on GitHub.com}
\label{new-branch}

\begin{figure}
	\centering
	\includegraphics[width=.9\textwidth]{./img/new-branch-1.png}
\end{figure}
\end{frame}

\begin{frame}{Create branch on GitHub.com}
\begin{figure}
	\centering
	\includegraphics[width=.9\textwidth]{./img/new-branch-2.png}
\end{figure}
\end{frame}

\begin{frame}{Create branch on GitHub.com}
\begin{figure}
	\centering
	\includegraphics[width=.9\textwidth]{./img/new-branch-3.png}
\end{figure}
\hyperlink{Create a branch}{\beamerreturnbutton{contents page}}
\end{frame}

\section{Appendix C : What are conflicts and how to solve them?}

\begin{frame}
	\frametitle{Conflict - setup}
	\begin{columns}[c]
		
		\column{.50\textwidth} % Right column and width
		\vspace{-.6cm}
		\begin{figure}
			\centering
			\includegraphics[width=.9\textwidth]{./img/conflict-network-setup.png}
		\end{figure}
		\vspace{-.3cm}
		\begin{figure}
			\centering
			\includegraphics[width=.9\textwidth]{./img/conflict-img-setup.png}
		\end{figure}
		
		\column{.50\textwidth} % Left column and width

		First we create a branch where we make one edit. 
		Changing \texttt{glitters} to \texttt{shines}
		
	\end{columns}
\end{frame}

\begin{frame}
	\frametitle{Conflict - commit}
	\begin{columns}[c]
		
		\column{.50\textwidth} % Right column and width
		\vspace{-.6cm}
		\begin{figure}
			\centering
			\includegraphics[width=.9\textwidth]{./img/conflict-img-commits-1.png}
		\end{figure}
		\vspace{-.3cm}
		\begin{figure}
			\centering
			\includegraphics[width=.9\textwidth]{./img/conflict-img-commits-2.png}
		\end{figure}
		
		\column{.50\textwidth} % Left column and width

		\vspace{-.6cm}
		\begin{figure}
			\centering
			\includegraphics[width=.9\textwidth]{./img/conflict-network-commits.png}
		\end{figure}
		
		\vspace{.5cm}
		
		In one feature branch of \textit{shine} one team member changes gold to silver,
		and in another feature branch one change gold to bronze.
			
	\end{columns}
\end{frame}

\begin{frame}
	\frametitle{Conflict - 1st merge}
	\begin{columns}[c]
		
		\column{.50\textwidth} % Right column and width
		\vspace{-.6cm}
		\begin{figure}
			\centering
			\includegraphics[width=.9\textwidth]{./img/conflict-network-1stmerge.png}
		\end{figure}
		
		\column{.50\textwidth} % Left column and width
		
		First we complete a brnach-pr-merge cycle for the \textit{shine-bronze} branch
		
		\vspace{.25cm}
		\footnotesize See PR at \trainingRepoURL{/pull/2}
		
	\end{columns}
\end{frame}

\begin{frame}
	\frametitle{Conflict - conflict}
	\begin{columns}[c]
		
		\column{.50\textwidth} % Right column and width
		\vspace{-.6cm}
		\begin{figure}
			\centering
			\includegraphics[width=.9\textwidth]{./img/conflict-img-conflict.png}
		\end{figure}
		\vspace{-.3cm}
		\begin{figure}
			\centering
			\includegraphics[width=.9\textwidth]{./img/conflict-img-commits-2.png}
		\end{figure}
		
		\column{.50\textwidth} % Left column and width
		
		\vspace{-.6cm}
		\begin{figure}
			\centering
			\includegraphics[width=.9\textwidth]{./img/conflict-network-conflict.png}
		\end{figure}
		
		\vspace{.5cm}
		
		We now have a conflict as in the \textit{shine-silver} branch
		the instructions are change \textit{gold} to \textit{silver},
		but in the \textit{shine} branch 
		it no longer says \textit{gold}, it says \textit{bronze}.
		
	\end{columns}
\end{frame}

\begin{frame}
	\frametitle{Conflict - warning}
	\begin{columns}[c]
		
		\column{.50\textwidth} % Right column and width
		\vspace{-.6cm}
		\begin{figure}
			\centering
			\includegraphics[width=.9\textwidth]{./img/conflict-img-cant-merge.png}
		\end{figure}
		\vspace{-.3cm}
		\begin{figure}
			\centering
			\includegraphics[width=.9\textwidth]{./img/conflict-img-resolve-1.png}
		\end{figure}
		
		\column{.50\textwidth} % Left column and width
		
		If the merge have conflict you will see a warning 
		and the \textit{Merge Pull Reques}t button is disabled. 
		Click the \textit{Resolve Conflict} button.
		
		\vspace{.5cm}
		
		When solving the conflict a code editor opens in your browser. 
		The conflicts starts at \texttt{<<<<<<<} and ends \texttt{>>>>>>>}. 
		And the parts that are from each branch is separated with a \texttt{=======} 
		
		\vspace{.25cm}
		\footnotesize See PR at \trainingRepoURL{/pull/3}
		
	\end{columns}
\end{frame}

\begin{frame}
	\frametitle{Conflict - resolve}
	

	\begin{minipage}[t][5cm][t]{\textwidth}
		\begin{figure}
			\centering
			\includegraphics[width=.55\textwidth]{./img/conflict-img-resolve-2.png}
		\end{figure}
	\end{minipage}
	
	\vspace{-1.5cm}
	
	\begin{minipage}[t][5cm][t]{\textwidth}
		
		Edit the file manually by using content in one or both branches until the conflict is solved. 
		Remember to remove the \texttt{<<<<<<<}, \texttt{>>>>>>>} and \texttt{=======},
		or they will become a part of your code.
		
		\vspace{.5cm}
		
		Click \textit{Mark as resolved} button when done. The keep reviewing the PR.
		
	\end{minipage}
\end{frame}

\begin{frame}{Version control}
	At the point of compiling this Beamer presentation, the most recent commit was:

	\commiturl

\end{frame}


\end{document}

\section{OLD SLIDES LOOKING FOR A NEW HOME}

\begin{frame}
	\frametitle{Step: Create an issue}
	\begin{columns}[c]
		
		\column{.35\textwidth} % Left column and width
		\begin{itemize}
			\setlength\itemsep{.5em}
			\item Issues is a far superior way to document tasks
			compared to emails or in-person meetings!
			\item They allow other team members to provide input
			on how to solve the task
			\item Very quick tasks that do not require discussion may not need an issue
		\end{itemize}
		
		\column{.65\textwidth} % Right column and width
		\vspace{-.75cm}
		\begin{figure}
			\centering
			\includegraphics[width=\textwidth]{./img/branch-pr-merge-cycle-old.png}
		\end{figure}
		
	\end{columns}
\end{frame}

\begin{frame}
	\frametitle{Un-draft the PR}
	\begin{columns}[c]
		
		\column{.6\textwidth} % Right column and width
		\vspace{-.75cm}
		\begin{figure}
			\centering
			\includegraphics[width=\textwidth]{./img/undraft-pr.png}
		\end{figure}
		
		\column{.4\textwidth} % Left column and width
		\begin{itemize}
			\setlength\itemsep{1em}
			\item Replace the \textit{work-in-progress} label
			with \textit{review-me}
			\item Assign one or several people to review your PR
			\item Notify the reviewer(s) using email
			or slack as GitHub notifications may be muted
			\item Click \textit{``Ready for review''} to remove draft status
		\end{itemize}
		
	\end{columns}
\end{frame}
